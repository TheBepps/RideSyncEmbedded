The most complete work on TEG is the paper of Tohidi, Holagh and Chitsaz ~\cite{TOHIDI2022117793} where the authors describe in details the working principle of TEG, the efficiency and the limitations of these devices, starting from thermodynamics principles then focusing on fabrication designs, modeling the TEG as a semiconductor device (also in ~\cite{Tian16}). One of the most relevant statements is that the \textbf{maximum output power of a TEG is achieved when the load resistance equals the internal resistance of the module}, useful when designing actual circuits or find the maximum efficiency of the TEG (related to the figure of merit). Then they explore the possible applications, especially in industrial processes and in the automotive sector (e.g., recovering waste heat from the engine).
\\
We found 3 examples of TEG characterization in the literature, which basically follow the same principles with minor variations.

\begin{itemize}
  \item Oswaldo Hideo Ando Junior, Nelson H. Calderon and Samara Silva de Souza developed in their work ~\cite{en11061555} a system to recover the energy from the waste heat of industrial processes. In these cases, the temperature difference is quite high, so the TEG can generate a significant amount of power. In particular they used a configuration of 10 TEG modules in series and 20 in parallel, reaching a \textbf{maximum power output of 29W} with a temperature difference of 80°C.
  \item The same strategy was employed by \cite{CARMO20112194} for characterization, namely measuring with an open circuit and a load resistor. The resulting model of the TEG was linear, both for the internal resistance and the Seebeck coefficient. Ultimately, the \textbf{SPICE model} of the TEG was obtained, thus enabling its use in simulations. 
  \item Finally, ~\cite{10209119} uses also the same approach, with the additional collection of data regarding thermal conductivity. Their methodology was based on steady-state principles, specifically \textbf{insulating the TEG with the heater} and logging the energy required to heat the TEG under load.   
\end{itemize}