% The conclusion goes here.
% Usually, the conclusion takes 15-20 lines (less than half a column) and should resume your told story.
% \\
% \\
% Context/Scenario  $\Rightarrow$  Challenges  $\Rightarrow$\\
%  $\Rightarrow$ What you have done $\Rightarrow$ \\
%  $\Rightarrow$ most important achievements, with key numbers. 
% \\
% Moreover, you could add short anticipation of advisable future works and directions to improve or to extend the project (beyond your task).  
% \\
% \\
% For more information, please refer to~\cite{IEEEexample:IEEEwebsite}. For \LaTeX~   information, you can have a look to~\cite{IEEEhowto:IEEEtranpage}  (Anyway \LaTeX~   is not mandatory).
% \\
% I wish you the best of success.

% In our project, we were able to find the internal resistance and Seebeck coefficient of the flexible fibre-based TEG. Due to a lack of the needed instrumentation, we were unable to collect the necessary data to determine the thermal conductivity. However, it can be obtained by measuring the efficiency. We know that the maximum efficiency of the TEG is achieved by matching impedance and internal resistance. Knowing the internal resistance, we can then find the maximum efficiency, from which we derive the figure of merit and the thermal conductivity.

We made measurements on three different types of TEG. The first two are classical TEG modules, and fitted results are in line with the online datasheets. The last one is a fibre-based TEG.\\

\vspace{5mm}

\begin{tabular}{ |p{2cm}|p{2.2cm}|p{2.2cm}|  }
    \hline
    TEG model & Seebeck & Internal resistance\\
    \hline
    TEG MAT & 0.034209 & 0.145028, 0.001342\\
    \hline
    TEG 12706 & 0.044441 & 2.299155, 0.083479\\
    \hline
    TEG VL25  & 0.007100 & 0.511499, 0.004171\\
    \hline
\end{tabular}

\vspace{5mm}

From the results obtained, the best TEG is the TEG MAT, having the highest Seeback coefficient and lowest internal resistance. Here is the link to the \href{https://github.com/gmazzucchi/tegc}{project repository}. In our project, we determined the internal resistance and Seebeck coefficient of the flexible fiber-based TEG. However, due to the unavailability of the requisite instrumentation, we were unable to collect the necessary data for determining the thermal conductivity. Nonetheless, this can be achieved by measuring the efficiency, as it is known that the maximum efficiency of the TEG is achieved by matching impedance and internal resistance. By determining the internal resistance, we can subsequently obtain the maximum efficiency, and thereby derive the figure of merit and the thermal conductivity.\\
In the future, it would be preferable to avoid setting temperatures manually. Instead, a cycle of temperatures should be set automatically by the microcontroller. This will allow for greater homogeneity in the dataset.

